\documentclass{article}

\begin{document}

Algorithmes CORDIC :


1) citer le chapitre ?

2) Quelle est la représentation des nombres utilisés sur une calculatrice ? Quels avantages et inconvénients pouvez-vous voir à ce genre de représentation ?

Dans une calculatrice, les nombres sont représentés en Binaire Code Decimal. Son principe : Les nombre sont représentas par des nombres décimaux et chaque chiffre est codé sur 4 bits.
Exemple :
12 en binaire est : 1100
1 en BCD est : 0001
2 en BCD est : 0010
donc 12 en BCD est : 0001 0010

Avanges :
- Multiplication et division par 10^n plus rapide : shift left ou droit de 4*n bits.
- Toutes les fonctions 'standard' peuvent se ramener aux quatre fonctions suivantes :  ln, exp, tan, arcta n
- Calculs plus rapides grâce aux algorithmes de CORDIC
- Précision de plus de 12 chiffres avec seulement une quinzaine de valeurs précalculées.

Inconvénients :
- Multiplication et division par 2^n plus lente que l'écriture binaire conventiennele : le shift de 1 bit ne multiplie ou divise plus.
- Pour un même nombre, prends plus de place en mémoire : besoin de nombre de chiffres * 4 bits contre log_2(nombre) +1
- Pour utiliser les algorithmes de CORDIC, on a besoin de certains valeurs précalculées.

3) Dans cette page, quatre algorithmes sont décrits pour calculer les fonctions trigonométriques et exponentielles. Quelle est la technique générale utilisée pour réaliser ces algorithmes ? En particulier, en quoi cette technique vous semble t’elle efficace lorsqu’elle est ramenée à une calculatrice ?

Technique générale utilisée pour réaliser ces algorithmes :
- On décompose le nombre en multiple (exp) ou somme (trigo) des valeurs précalculées
- Pour chaque valeur précalculée,

exploirent les dev limités à l'ordre 1
rendre x tres tres petit
produit partiel = gros produit
\end{document}
